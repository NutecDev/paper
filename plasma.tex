\documentclass[letterpaper, 11pt]{article}

\usepackage{graphicx}
%\usepackage{fullpage}
\usepackage{pdfpages}
\usepackage{color}
\usepackage[colorlinks=true,urlcolor=blue,citecolor=black]{hyperref}
\usepackage{url}
\usepackage[font=footnotesize,labelfont=bf]{caption}
%full name for appendix
%\usepackage[title]{appendix}
\usepackage{float}
%\usepackage{parskip}
%for code
\usepackage{listings}
%for math
\usepackage{amsmath}

\usepackage[margin=1.25in]{geometry}

%Pargraph line breaks
%\setlength{\parskip}{1em}
%\setlength{\parindent}{0pt} % Default is 15pt.


\linespread{1.1}
\setlength{\emergencystretch}{3em}


%opening
\title{\LARGE Plasma: Scalable Autonomous Smart Contracts}
\author{
                Joseph Poon\\
                \footnotesize\href{mailto:joseph@lightning.network}
                        {\nolinkurl{joseph@lightning.network}}
			\and
                Vitalik Buterin\\
                \footnotesize\href{mailto:vitalik@ethereum.org}
                        {\nolinkurl{vitalik@ethereum.org}}
			%\and
		%Christian Reitwie{\ss}ner\\
                %\footnotesize\href{mailto:chris@ethereum.org}
                %        {\nolinkurl{chris@ethereum.org}}
        }   

\date{\today\\\small WORKING DRAFT\\\small https://plasma.io/}


\begin{document}

\maketitle


\begin{abstract}
	Plasma is a proposed framework for incentivized and enforced execution
	of smart contracts which is scalable to a significant amount of state
	updates per second (potentially billions) enabling the blockchain to be
	able to represent a significant amount of decentralized financial
	applications worldwide. These smart contracts are incentivized to
	continue operation autonomously via network transaction fees, which is
	ultimately reliant upon the underlying blockchain (e.g. Ethereum) to
	enforce transactional state transitions.

	We propose a method for decentralized autonomous applications to scale
	to process not only financial activity, but also construct economic
	incentives for globally persistent data services, which may produce 
	an alternative to centralized server farms.

	Plasma is composed of two key parts of the design: Reframing all
	blockchain computation into a set of MapReduce functions, and an
	optional method to do Proof-of-Stake token bonding on top of existing
	blockchains with the understanding that the Nakamoto Consensus
	incentives discourage block withholding.

	This construction is achieved by composing smart contracts on the main
	blockchain using fraud proofs whereby state transitions can be enforced
	on a parent blockchain. We compose blockchains into a tree hierarchy,
	and treat each as an individual branch blockchain with enforced
	blockchain history and MapReducible computation committed into merkle
	proofs. By framing one's ledger entry into a child blockchain which is
	enforced by the parent chain, one can enable incredible scale with
	minimized trust (presuming root blockchain availability and
	correctness).

	The greatest complexity around global enforcement of non-global data
	revolves around data availability and block withholding attacks, Plasma
	has mitigations for this issue by allowing for exiting faulty chains
	while also creating mechanisms to incentivize and enforce continued
	correct execution of data.

	As only merkleized commitments are broadcast periodically to the root
	blockchain (i.e. Ethereum) during non-faulty states, this can allow for
	incredibly scalable, low cost transactions and computation. Plasma
	enables persistently operating decentralized applications at high
	scale.
\end{abstract}

\section{Scalable Multi-Party Computation}

With blockchains, the solution for enforcing correctness has generally been
having every participant validate the chain themselves. To accept a new block
requires one to fully validate the block to ensure correctness. Many efforts to
scale blockchain transactional capacity (e.g. Lightning
Network\cite{Lightning}) requires using time commitments to build a fidelity
bond, (an assert/challenge agreement) so that the asserted data must be subject
to a dispute period for participants on the blockchain to enforce the state.
This assert/challenge construction allows one to assert a particular state is
correct, and if the value is incorrect, then a dispute period exists where
another observer can provide a proof challenging that assertion before a
certain agreed time. In the event of fraudulent or faulty behavior, the
blockchain can then penalize the faulty actor. This creates a mechanism for
participants to be encouraged to enforce if-and-only-if the incorrect state is
asserted. By having this assert/challenge-proof construction, interested
participants can be able to assert ground truths to non-interested participants
on the root blockchain (e.g.
Ethereum\cite{ethereumwhite}\cite{ethereumyellow}).

This structure can be used not only for payments, but extended to computation
itself so that the blockchain is the adjudication layer for contracts. However,
the presumption would be that all parties are participants in validating the
computation. In Lightning Network, for example, the construction makes it so
that one can establish commitments to computing contract state (e.g. with
pre-signed tree of multisignature transactions of conditional state).

These constructions allow for highly powerful computation at scale, however
there are some issues which require the summation of a lot of external state
(i.e. summation of entire systems/markets, computation of a large amount of
shared/incomplete data, large number of contributors). This form of commitment
to multiparty off-chain state ("state channels"\cite{raiden}) requires
participants to fully validate the computation, or else there are significant
amount of trust established in the computation itself, even in single-round
games. Additionally, there is usually a presumption of "rounds" whereby the
execution path must be completely unrolled before contract initiation, which
gives participants the opportunity to exit and force expensive computation
on-chain (as it is not possible to prove which party is halting).

Instead, we seek to design a system whereby computation can occur off-blockchain
but ultimately enforcible on-chain which is scalable to billions of computations
per second with minimal on-chain updates. These state updates occur across an
autonomous set of proof-of-stake validators who are incentivized towards correct
behavior enforced by fraud proofs, which allow for computation to occur without
a single actor being able to easily halt the computation service. This needs to
be able to minimize issues around the data availability problem (i.e. block
withholding), minimizing the state updates in the root blockchain necessary in
the event of byzantine actors to prevent risk-discounted transaction fees on the
root chain, and a mechanism to enforce state changes.

Similar to the Lightning Network, Plasma is a series of contracts which runs on
top of an existing blockchain to ensure enforcement while ensuring that one is
able to hold funds in a contract state with net settlement/withdrawal at a later
date.

\section{Plasma}

Plasma is a way to do scalable computation on the blockchain with the structure
of creating economic incentives to autonomously and persistently operate the
chain without active state transition management by the contract creator. The
nodes themselves are incentivized to operate the chain.

Additionally, significant scalability is achieved by minimizing the funds
represented in a spend from a contract to a single bit in a bitmap, so that
one transaction and signature represents a payment coalesced with many
participants. We combine this with a MapReduce\cite{mapreduce} framework to be
able to construct scalable computation enforced by bonded smart contracts.

This construction allows one to be able to have externalized parties hold funds
and compute contracts on one's behalf similar to a miner, but Plasma instead
runs on top of an existing blockchain so that one does not need to create
transactions on the underlying chain for every state update (including adding
new users' ledger entries), with minimal data on-chain for coalesced state
updates.

%Diagram - Plasma-Summary-Many-Chains.dia
\begin{figure}[H]
	\makebox[\linewidth]{
		\scalebox{0.75}{
		\includegraphics[width=\linewidth]{figures/Plasma-Summary-Many-Chains.pdf}
		}
		}
	\caption{
		Anyone can create a custom Plasma chain for smart contract
		scalability for many different use cases. Plasma is a series of
		smart contracts which allows for many blockchains within a root
		blockchain. The root blockchain enforces the state in the Plasma
		chain. The root chain is the enforcer of all computation
		globally, but is only computed and penalized if there is proof
		of fraud. Many Plasma blockchains can co-exist with their own
		business logic and smart contract terms. In Ethereum, Plasma
		would be composed of EVM smart contracts running directly on
		Ethereum, but only processing tiny commitments which can
		represent an incredibly large amount of computation and
		financial ledger entries in non-Byzantine cases.
		}
\end{figure}


Plasma is composed of five key components: An incentive layer for persistently
computing contracts in an economically efficient manner, structure for arranging
child chains in a tree format to maximize low-cost efficiency and net-settlement
of transactions, a MapReduce computing framework for constructing fraud proofs
of state transitions within these nested chains to be compatible with the tree
structure while reframing the state transitions to be highly scalable, a
consensus mechanism which is dependent upon the root blockchain which attempts
to replicate the results of the Nakamoto\cite{nakamoto} consensus incentives,
and a bitmap-UTXO commitment structure for ensuring accurate state
transitions off the root blockchain while minimizing mass-exit costs. Allowing
for exits in data unavailability or other Byzantine behavior is one of the key
design points in Plasma's operation.

\subsection{The Plasma Blockchain, or Externalized Multiparty Channels}

We propose a method whereby multiparty off-chain channels can hold state on
behalf of others. We call this framework a Plasma blockchain. For funds held in
the Plasma chain, this allows for deposit and withdrawal of funds into the
Plasma chain, with state transitions enforced by fraud proofs. This allows for
enforcible state and fungibility since one is able to deposit and withdraw, with
accounting of the Plasma block matching the funds held in the root chain (Plasma
is not designed to be compatible with fractional reserve banking designs).

%Diagram - Plasma-Summary-Externalized-State.dia
\begin{figure}[H]
	\makebox[\linewidth]{
		\scalebox{1}{
		\includegraphics[width=\linewidth]{figures/Plasma-Summary-Externalized-State.pdf}
		}
		}
	\caption{
		Plasma blockchains are a chain within a blockchain. The system
		is enforced by bonded fraud proofs. The Plasma blockchain does
		not disclose the contents of the blockchain on the root chain
		(e.g. Ethereum). Instead, blockheader hashes are submitted on
		the root chain and if there is proof of fraud submitted on the
		root chain, then the block is rolled back and the block creator
		is penalized. This is very efficient, as many state updates are
		represented by a single hash (plus some small associated data).
		This update can represent balances which are not represented on
		the root blockchain (Alice does not have her ledger balance on
		the root chain, her ledger is on the Plasma chain, and the
		balance in the root chain is representing a smart contract
		enforcing the Plasma chain itself). Gray items are old blocks,
		black is the most recent block that has been propagated and
		committed on the root chain.
	}
\end{figure}

Incredibly high amount of transactions can be committed on this Plasma chain
with minimal data hitting the root blockchain. Any participant can transfer
funds to anyone, including transfers to participants not in the existing set of
participants. These transfers can pay into and withdraw (with some time delay
and proofs) funds in the root blockchain's native coin(s)/token(s).

Plasma allows one (or a network of participants in a proof-of-stake network) to
be able to manage a blockchain without a full persistent record of the ledger on
the root blockchain and without giving custodial trust to the 3rd party or
parties. In the worst case, funds are locked up and time-value is lost with
mass-exits on the blockchain.

We construct a series of fraud proofs as smart contracts\cite{smartcontracts}
on the root blockchain which enforce state in this channel so that attempts at
fraud or non-Byzantine behavior can be slashed.

These fraud proofs enforce an interactive protocol of fund withdrawals. Similar
to the Lightning Network, when withdrawing funds, the withdrawal requires time
to exit. We construct an interactive game whereby the exiting party attests to a
bitmap of participants' ledger outputs arranged in an UTXO model which
requests a withdrawal. Anyone on the network can submit an alternate *bonded*
proof which attests whether any funds have already been spent. In the event this
is incorrect, anyone on the network can attest to fraudulent behavior and slash
the bonds to roll back the attestation. After sufficient time, the second
*bonded* round allows for the withdrawal to occur, which is a bond on state
*before* a committed timestamp. This allows for a withdrawal en masse so that a
faulty Plasma chain can be rapidly exited. In coordinated mass withdrawal
events, a participant may be able to exit with less than 2-bits of block space
consumed on the parent blockchain (i.e. root Ethereum on-chain in worst case
scenarios).

In the event of a block withholding attack, participants can rapidly and cheaply
do a mass-exit, with substantial cost savings versus other previous off-chain
proposals. Additionally, this does not place any trust in a coalition of
validator nodes (Sidechain Functionaries, Fishermen).

%Diagram - Plasma-Summary-Exit.dia
\begin{figure}[H]
	\makebox[\linewidth]{
		\scalebox{1}{
		\includegraphics[width=\linewidth]{figures/Plasma-Summary-Exit.pdf}
		}
		}
	\caption{
		Exit of funds in the event of block withholding. The red block
		(Block \#4) is a block which is withheld and committed on the
		root chain, but Alice is not able to retrieve Plasma block \#4.
		She exits by broadcasting a proof of funds on the root
		blockchain and her withdrawal is processed after a delay to
		allow for disputes.
		}
\end{figure}

Similar to how closing out Lightning is an interactive mechanism between two
participants to enable enforcible infinite payments between themselves, this
allows for an interactive mechanism between n participants. The primary
difference is that not all participants need to be online to update state, and
the participants do not need a record of entry on the root blockchain to enable
their participation -- one can place funds on Plasma without direct interaction
on-chain, with minimal data to confirm transactions when constructing these
Plasma chains in a tree format.

\subsection{Enforcible Blockchains in Blockchains}

%Diagram - Plasma-Summary-Tree.dia
\begin{figure}[H]
	\makebox[\linewidth]{
		\scalebox{0.7}{
		\includegraphics[width=\linewidth]{figures/Plasma-Summary-Tree.pdf}
		}
		}
	\caption{
		Plasma composes blockchains in a tree. Block commitments flow
		down and exits can be submitted to any parent chain, ultimately
		being committed to the root blockchain.
		}
\end{figure}

We construct a mechanism similar to the court system. If Lightning Network uses
an adjudication layer for payments which is ultimately enforcible on the root
blockchain, we create a system of higher and lower courts to maximize
availability and minimize costs in non-Byzantine states. If a chain is
Byzantine, it has the option of going to any of its parents (including the root
blockchain) to continue operation or exit with the current committed state.
Instead of enforcement of an incrementing nonce state (via revocations), we
construct a system of fraud proofs to enforce balances and state transitions of
these chain hierarchies.

In effect, we are able to create state transitions which are only periodically
committed to parent chains (which then flows to the root blockchain). This
allows for incredible scale of computation and account state, as we are able to
only submit raw data to parent (or root) chain in Byzantine conditions. Recovery
from partially Byzantine conditions are cost-minimized since one can go to a
parent Plasma chain to enforce state.

This child blockchain runs on top of a root blockchain (e.g. Ethereum) and from
the root blockchain's perspective, is only seeing periodic commitments with the
tokens bonded in the contract for enforcement of the Proof-of-Stake consensus
rules and business logic of the blockchain.

This has significant advantages in maximizing block availability and minimizing
stake for validation of one's coins. However, since not all data is being
propagated to all parties (only those who wish to validate a particular state),
parties are responsible for monitoring the particular chain they are interested
in periodically to penalize fraud, as well as personally exiting the chain
rapidly in the event of block withholding attacks.

%Diagram - Plasma-Summary-Tree-Failure.dia
\begin{figure}[H]
	\makebox[\linewidth]{
		\scalebox{1.0}{
		\includegraphics[width=\linewidth]{figures/Plasma-Summary-Tree-Failure.pdf}
		}
		}
	\caption{
		The faulty blockchain (shaded in red) is routed around by
		broadcasting a commitment to its parent Plasma/root chain(right
		dotted blue line). Participants in the 3rd depth Plasma chain
		do a mass migration to another chain together (left blue dotted
		line) after some period of time.
		}
\end{figure}

This construction, in non-Byzantine environments, coalesces the tree of
blockchain states and update all child Plasma chains. An entire set of updates
across all chains can be attested to in a 32-byte hash with a signature.

\subsection{Plasma Proof-of-Stake}

While it's fairly interesting to be able to hold fund on behalf of others with a
single validator, we propose a method whereby a single party can enforce state
with a set of validators, often in a proof-of-stake framework requiring either
ETH bonding, or bonding in a token (e.g. ERC-20).

The consensus mechanism for this proof of stake system, is again, enforced in an
on-blockchain smart contract.

We attempt to replicate the incentives around the Nakamoto Consensus, but using
Proof-of-Stake bonds. We believe that one of the more useful incentive
mechanisms constructed as a result of the Nakamoto mechanism is that there is
incredible incentive to minimize block withholding attacks. This is since
leaders are only probabilistically elected. Leaders are probabilistically known
over time (in the original implementation it was 6 confirmations). When one
finds a block, one is fairly sure they are likely the leader, but they are not
yet certain if they are the leader. To ensure that they are the leader, they
propagate their blocks to all participants on the network to maximize their
odds. We believe this is a significant if not the key contribution with the
Nakamoto mechanism and attempt to replicate this incentive.

Proof-of-stake coalitions face this issue since it's possible if one does
straight leader election, block withholding attacks by majority cartels (also
generalized as the "data availability problem") become magnified.

We can mitigate this in Plasma Proof-of-Stake by allowing stakeholders to
publish on the root blockchain or parent Plasma chain which contains a committed
hash of their new block. Validators only build upon blocks which they have fully
validated. They can build upon blocks in parallel (to encourage maximum
information sharing). We create incentives for validators to represent the
past 100 blocks to match the current staker ratio (i.e. if one stakes 3 percent
of the coins, they should be 3 percent of the past 100 blocks), by rewarding
more transaction fees to be paid out to accurate representation. Excess fees
(due to suboptimal behavior by stakers) goes to a pool to pay out fees in the
future. A commitment exists in every block which includes data from the past 100
blocks (with a nonce). The correct chain tip is the chain with summed weight of
the highest fees. After a period of time, the blocks are finalized.

%Diagram - Plasma-Summary-PoS.dia
\begin{figure}[H]
	\makebox[\linewidth]{
		\scalebox{0.75}{
		\includegraphics[width=\linewidth]{figures/Plasma-Summary-PoS.pdf}
		}
		}
	\caption{
		Presume that Alice, Bob, and Carol are 3 validators with equal
		amount of weight. They collectively are incentivized to build a
		round-robin structure for maximum returns. These commitments are
		submitted to the parent/root chain. Chain tip is contingent on
		maximum weight score by correct distribution of blocks over n
		periods (blue is the current candidate chain tip, red is an
		orphan). Suboptimal chain tips has any excess fees go into a
		pool for future validators with correctness above some
		threshold (e.g. 90\%). After some n periods it's presumed that
		the blue chain tip is finalized.
		}
\end{figure}

This encourages participants to participate and replicates the 51\% attack
assumptions in the Nakamoto consensus. In the event a chain is attacked via
block withholding or other Byzantine behavior, the non-Byzantine participants
conduct a mass compact withdrawal on the parent/root blockchain. If bonds for
the highest parent Plasma chain are in the form of tokens, it is likely the
value of the token will significantly devalue as a result of the mass exit.

\subsection{Blockchains as MapReduce}

\begin{center}
	blockchain : git :: Plasma : Hadoop (MapReduce)
\end{center}

By constructing computation in a MapReduce format, it is also easy to design
computation and state transitions in a hierarchical tree.

MapReduce gives a framework for high scale computation across thousands of
nodes. The blockchain faces similar issues in meeting computational scale, but
has additional requirements in generating proofs of computation.

%Diagram - Plasma-Summary-MapReduce.dia
\begin{figure}[H]
	\makebox[\linewidth]{
		\scalebox{1}{
		\includegraphics[width=\linewidth]{figures/Plasma-Summary-MapReduce.pdf}
		}
		}
	\caption{
		The blue is messages passed in the parent block to the children.
		The children must commit to the parent block within some n
		number of blocks or else face chain halting. The block data
		distributes work to the children who are committing to
		computation. The 3rd level child does the computation and
		returns a wordlist (e.g. 3 occurrences of the word "Hello", 2
		occurrences of the word "World" in the chapter they are
		responsible for computing). The wordlist data is returned to the
		parent as part of the commitment, wordlists are combined
		together from the children and submitted to the parent,
		ultimately completing a global wordlist (e.g. the entire corpus
		contains 100 instances of "Hello" and 150 instances of the word
		"World"). This creates economically enforcible computation at
		scale, with only one block header/hash committed on the root
		chain to encompass very high amount of data and work. It is only
		if a block is faulty that proof of invalidity is published,
		otherwise extremely minimal amounts of data is submitted on the
		root chain periodically.
		}
\end{figure}

We propose a method whereby the map phase includes commitments on data to
computation as input, and in the reduce step includes a merkleized proof of
state transition when returning the result. The merkleized state transition is
enforced via fraud proofs constructed on the root blockchain. It is also
possible to construct a zk-SNARKs proof of state transitions. For some
computational constructions, a bitmap on state transitions may also be
necessary in the reduce step (therefore more than one bit can be used per
UTXO/account for these use cases).

Our construction enables incredible high-scale computation, with time or speed
tradeoffs. These tradeoffs produce a network where nodes assert computation and
participants are responsible for verifying them. This does not produce a system
whereby one can completely outsource computation without trust, it enables the
ability to compress computation into bonded proofs. These bonded proofs
encourage participants to only attest to things honestly. This again, follows
the narrative in the Lightning Network, whereby if a tree falls in the forest
and nobody listens to it, it presumes that it doesn't matter whether it makes a
sound or not. Similarly, if no one is watching/enforcing the computation, it's
presumed to be correct, or it simply doesn't matter what the result may be.
Computation can be watched by any participant in open networks, but participants
who hold balances and/or require correct computation will periodically watch the
chain to ensure correctness. The scaling benefit comes from removing the
requirement to watch the chains one is not economically impacted by, one should
watch the chains where one wishes to enforce correct behavior. Behavior on other
Plasma chains can be netted together as part of the reduce step so that the
computation which affects one is expressed in a minimal state. For example, for
a decentralized exchange, one doesn't care about which counterparties put in
what order, they only need to see one coalesced orderbook, so one only needs to
observe all other chains as a single counterparty, whereas one's own chain is
fully validated to enforce transactions and order fills to the correct person
(including oneself). Another example is one can construct a BBS on a tree of
Plasma chains, and one doesn't need to receive updates on the topics one doesn't
care about.

%Diagram - Plasma-Summary-MapReduce-Watch.dia
\begin{figure}[H]
	\makebox[\linewidth]{
		\scalebox{0.75}{
		\includegraphics[width=\linewidth]{figures/Plasma-Summary-MapReduce-Watch.pdf}
		}
		}
	\caption{
		One only needs to watch the data which one wants to enforce. If
		economic activity or computation occurs on other Plasma chains
		which is not necessary to enforce (gray), one can treat all
		other chains as a single counterparty. E.g. in a Plasma
		decentralized exchange, one just needs to watch the chains which
		affects one's commitments (bolded blue).
		}
\end{figure}

\subsection{A Description of Economic Incentives around Persistent Decentralized
Autonomous Blockchains}

We propose a structure whereby one can create economic incentives to
persistently keep a child blockchain running. For state which does not require
significant complexity or reliance in state transitions, the native token (e.g.
ETH for Ethereum) can be used for bonding of state. However, for complex
contracts, there may be significant incentives to continue operating the chain
due to incentives around ensuring liveness and order fairness of the
system\cite{fredblog}\cite{navalblog}.

Every Plasma chain is represented by a set of contracts. These contracts enforce
the consensus rules of the chain, and fraud causes significant penalties to be
applied if the fraud proof can be produced.

However, to incentivize avoidance of Byzantine states, especially around
correctness and liveness, it may be ideal to create a token per contract. This
token represents the network effects in operating the contract, and creates an
incentive to maximize security of this contract. As the Plasma chain requires
the token to secure the network in a Proof-of-Stake structure, stakers are
disincentivized against Byzantine behaviors or faults as that would cause a loss
in value of the token. The role of the token is to ensure there is costs
localized to the validators if they act faulty via value declines in the token.

With simple contracts and business logic such as a basic contract account
holding funds on behalf of its users, an Ethereum bond can represent a stake in
the Plasma chain.

The stakes who put up bonds (whether it be a token or ETH) have incentives to
continue operating the network as they receive transaction fees for operating
the network. These transaction fees are then paid out to the stakers of the
network which encourages non-Byzantine behavior, and creates long-term value for
the token.

Since the stakers have an incentive to continue operating this network to
collect transaction fees, they will persistently run the chain and are bound by
the fraud proofs defined in the contracts in the root blockchain.

\section{Design Stack and Smart Contracts}

Historically, many people believe that the blockchain is best applied towards
transactional payments as a gross settlement system. However, it's understood
that a gross settlement system has difficulty scaling. Net settled designs such
as the Lightning Network, a payment channel network, changes the structure to
allow for nearly unlimited payments between participants. Transactional capacity
is increased dramatically as channels are net-settled on the blockchain.
Payments can be routed across a network of these channels.

This structure additionally allows for effectively instantaneous payments. This
is instrumental for not only payments which require a high degree of time
sensitivity, but also for contracts as well.

Plasma is not designed to reach assured finality rapidly, even though
transactions are confirmed in the child chains rapidly, it requires it to be
finalized on the underlying root blockchain. Channels are necessary to be able
to have rapid local finality of payments and contracts (enforcible on-chain).

In smart contracts, there is an issue of the "free option problem" whereby the
receiver (second or last signer) of a smart contract offer is needed to sign and
broadcast the contract in order to enforce it -- during that time the receiver
of the contract may treat it as a free option and refuse to sign the contract if
the activity does not interest them. This is exacerbated as smart contracts are
most effective when dealing with counterparties who are untrusted (as that
creates minimization in counterparty risk and thereby information costs).

Plasma does not solve this problem on its own, as there are no guarantees of
atomicity with the first and second signature step for interactive protocols in
blockchains.

With Lightning (including Lightning on top of Plasma), it's possible to do
incredibly rapid updates with reasonable sense of localized finality. Instead
of having a single payment which gives optionality to the last party, a payment
can instead be split into many small payments. This minimizes the free option
to the amount per split fraction. Since the second party of the smart contract
only has the free option on the amount in the split fraction, the value of the
free option is minimized.

Within the above use cases, it's possible that Lightning may be a primary
interface layer for rapid financial payments/contracts on top of Plasma, as
Plasma allows for ledger updates with minimal root chain state commitments.

%Diagram - Blockchain/Plasma/Lightning Stack 
\begin{figure}[H]
	\makebox[\linewidth]{
		\scalebox{0.2}{
		\includegraphics[width=\linewidth]{figures/Plasma-LN-Stack.pdf}
		}
		}
	\caption{
		At the root is the blockchain, which is the adjudication layer
		for contracts and payments. The contracts themselves are located
		on the root blockchain. The Plasma chain contains the current
		ledger state which can be settled and redeemed on the
		root blockchain. Fraud proofs exist to allow for funds to be
		redeemed. Plasma represents a nested set of Plasma chains to
		create venues to withdraw funds in a scalable way with minimal
		blockchain transactions. On top is the Lightning Network, which
		allows for instantaneous payments across Plasma and Block
		Chains.
	}
\end{figure}

\subsection{The Most Significant Problem in Sharding is Information}

With sharded data sets, there is a significant risk of individual shards to
refuse to disclose information. It would thereby be impossible to produce fraud
proofs.

We attempt to resolve this using 3 strategies:

\begin{enumerate}
	\item 
		A new Proof-of-Stake mechanism which encourages block propagation.
		The underlying mechanism does not entirely rely upon correct
		functionality of incentives. However, this should significantly
		decrease faulty behavior.

	\item 
		Significant withdrawal delays which allow for accurate withdrawal
		proofs. Individuals don't need to watch the blockchain that
		frequently and fraud on higher plasma chains can be prevented on
		the root blockchain by any honest actor on the same plasma chain
		as the user. In the event of block withholding, plasma chains
		can immediately lock up funds via a proof, preventing an
		attacker from submitting fraudulent withdrawal proofs. In the
		event the attacker attempts to withdraw funds above their limit
		and more funds are locked, the attacking plasma chain loses
		their deposit.

	\item 
		Creating child chains whereby transactions can be propagated in
		any parent chain. For this reason, participants on networks will
		desire to submit transactions to deep child chains. This creates
		economic efficiency for smaller balances which do not have the
		economic ability to pay high transaction fees on the root
		blockchain and therefore moving funds can be achieved with many
		small balances. People are therefore encouraged to create deeply
		nested child chains which represent significant value. Note that
		there is some presumption about reputation around chain
		selection for individuals holding very small balances which
		cannot be on the root blockchain transaction fees, however is
		mitigated by having deeply nested chains. This security model is
		the key novelty of plasma chains.
\end{enumerate}

\section{Related Work}

Some related projects propose a merkle tree with reduction steps as proof of
computation, however this proposal is primarily around data availability and
encouraging cost minimization around fraud proofs, with a protocol to manage
these via an economically incentivized persistent sharded group of chains.

Other related projects propose a system of child blockchains, but have
significant differences in approach.

Plasma uses a merkleized proof to enforce child chains.

\subsection{TrueBit}

Plasma shares a great degree of similarity in the reliance of fraud proofs as
TrueBit\cite{truebit}. The fraud proof construction is similar to TrueBit, and
nearly all the work by TrueBit can directly apply to Plasma, especially the
work around merkleized proofs of state transitions.

TrueBit design allows for compact proofs to be created to submit to the Ethereum
blockchain, which is necessary for Plasma, so nearly all of the heavy lifting
done by the TrueBit paper and team is directly applicable in this design. The
use of the Verification Game, which generates merklized proofs provides
increased benefits with reducing the scale of computation. Similar assumptions
as TrueBit applies, namely that computational state must be computable and
broadcastable online (large pieces of data must be split in multiple rounds),
data availability problems needs to be mitigated, failure must be disclosed. We
attempt to mitigate these problems, especially the latter two.

The primary aspect which Plasma attempts to build upon TrueBit is the notion of
multiparty participants which need to compute on shared state. For example, a
set of participants only care about a subset of the data and computation and
needs to only compute aspects related to themselves (e.g. BBS or exchange). We
also attempt to mitigate the issue of enforcement of computational rounds via
off-chain venues of enforcement.

\subsection{Blockchain Sharding}

Current work on blockchain sharding\cite{ethsharding} uses similar
techniques and goals, e.g. Ethereum Sharding proposal. This construction may be
compatible as a higher layer. If the root blockchain is sharded, then the
Plasma chain can run on top of this for greater scalability and other benefits.
This can also be a testbed for different sharding techniques as there are no
consensus changes necessary in Ethereum and other rich stateful blockchains to
begin basic operation.

\subsection{Federated Sidechains}

Plasma is not a federated sidechain\cite{sidechains} as it does not rely upon
the federation for honest activity, nor does it rely entirely upon trusted
actors to enforce state inside the chain. Plasma also externalizes ledger state
to another blockchain allowing the use of the same coin/token, however it does
enforcible verification if the fraud proof is available. Plasma does not rely
upon a strong federation of actors, which requires significant underwriting
risk on the correctness of these actors, for this reason it is not a
Federated-Pegged Sidechain. 

Drivechains\cite{drivechains} shares similarity to federated sidechains, except
the validators are an unknown, changing set of participants (miners), with
greater decentralization.

\subsection{Merge-Mined Blockchain}

Examples include Namecoin, which create concurrent blocks with the parent
blockchain\cite{mergedmining}. This presumes full validation of the blockchain,
thus does not provide scalability benefits. Extension blocks are an example of
a merge-mined chain which allows for funds to move between the primary
blockchain and the merge-mined chain (with an enforcement mechanism of the full
set of miners as a consensus rule on the root chain). Merge-mined chains permit
for new consensus rules and election of users to validate only the chains they
care about, but the miners/validators must validate everything. The goal of
Plasma is to ensure that only users and miners need to validate the chains
relevant to themselves.

\subsection{Treechains}

Treechains\cite{treechains} proposes tree structures blockchains which are
validated in the child blocks using Proof of Work. The root chain has the
summed proof of work of all child blockchains. Lower down the stack has higher
security, but higher along the stack may or may not depending on the level of
validation and work. While treechain's topology is in a tree structure, its
structure is dependent upon mining security being summed via the branches. The
security model has lower security on the leaves, as it is secured by
Proof-of-Work. Plasma is the reverse with the mining done with full security
only on the root, with the security and proofs flowing from the root. Similar
work is in building proofs of blocks seen in a tree formation.

\subsection{zk-SNARKs and zk-STARKs}

Non-interactive proofs of computation allow for one to have significant
benefits in scalable computation\cite{snarks}. zk-SNARKs/STARKs and other forms
of non-interactive compact proofs is complimentary to Plasma. A proof can be
provided along with the result of the merklized computation. Additionally,
benefits exist to reduce systemic attacks when holding small balances on a
child plasma chain. There has already been research in SNARKs for MapReduce
functionality\cite{snarksmr} and we hope this leverages on that research and
Plasma extends it by making the proofs orderable and enforcible within a set of
blockchains.

Further benefits include proofs of computation which allow for faster syncing
and verifying of chains themselves. Note that zk-SNARKs does not solve the issue
around data availability, just reduces the amount of data requirements and
computation. This is especially useful as a replacement or complement for any
assert/challenge time-based mechanisms. zk-SNARKs can be helpful as defense in
depth. If the last line of defense is using the blockchain without fancy
cryptography, the second line of defense can be zk-SNARKs, and the first line of
defense is trusted computing hardware.

Withdrawals from Plasma chains could be secured by zk-SNARKs which gives the
benefit of optionally not requiring the bitmap, which may allow for very small
balances to be transferred.

\subsection{Cosmos/Tendermint}

%TODO: Summary of cosmos
Cosmos\cite{cosmos} arranges blockchains in a Cosmos "Hub" and has child blockchains "Zones"
validated over a proof of stake system. Significant similarity with the
construction of child blockchains exist, however Plasma is reliant upon
construction fraud proofs to enforce state in child chains and is genericized
to be applicable to many chains. The proof of stake construction for Cosmos
presumes a 2/3 honest majority of validators, including validators of its
Cosmos Zone.

\subsection{Polkadot}

Polkadot\cite{polkadot} also constructs a structure for a hierarchy of
blockchains. There is some similarity with the design of Polkadot. Instead of a
structure with "fishermen" validators ensuring block accuracy, we construct a
series of child blockchains which enforce state via merkle proofs. The Polkadot
construction is reliant on the child blockchains ("parachains") state and
information availability being enforced by the fishermen.

\subsection{Lumino}

Lumino\cite{lumino} is a design for an EVM contract with compressed updates on
the blockchain. This allows participants to only update minimal committed
state. Plasma's output management design takes things further with only a
single bit denoting a particular output. This allows for rapid, low cost
coordination around mass withdrawal in the event of child Plasma chain failure.

\section{Multiparty Off-Chain State}

The goal is to construct a method whereby participants can hold funds in the
native underlying coin/token of the blockchain, without significant on-chain
state. Plasma begins to blur the line between on-chain and off-chain (e.g. are
shards on-chain or off-chain?).

There are two common issues in efforts to establish off-blockchain multiparty
channels. The first is the need to do synchronized state update amongst all
participants when there needs to be an update on the system (or otherwise make
tradeoffs on availability of global state updates) and therefore must be online.
The second is that adding and removing participants in the channel require a
large on-blockchain update, enumerating all participants which are added and
removed.

It would instead be preferable to construct a mechanism whereby many
participants can be added and removed without significant root chain state
updates and internal state updates are possible without all parties
participating, they only need to participate if their balances are being
adjusted or if Byzantine behavior is detected.

The general construction is a child blockchain which allows for holding
balances represented in a smart contract on the root blockchain (e.g.
Ethereum). The balances of the smart contract are represented and allocated to
the balances of finalized blocks in the child Plasma blockchain. This allows
one to hold the native coin in the child blockchain with full representation of
balances on the root blockchain, allowing for withdrawals after a dispute
mediation period.

In order to achieve this, we construct a UTXO (Unspent Transaction Output) model
for the ledger. While this is not an explicit requirement, it is easier to
reason about with rapid withdrawals. The rationale for a UTXO model is that it
is easy to compactly represent whether a particular state has been spent or not.
This can be represented within a trie for merkleized proofs, and as a bitmap for
a compact representation parsable by others. In other words, the smart contracts
are held in accounts on the root chain, but the Plasma chain maintains a UTXO
set of balances for the allocation of the balances held in the root chain
account. For child chains which do not have significant requirements around
state transitions, it is possible to use an account model for more complex or
frequent state transitions, however there is more reliance on block space
availability in the parent blockchain(s).

For now, one can presume a single block leader which selects a block of the
child Plasma chain. It is possible to construct this as a Proof of Stake set or
a named preset n-of-m validators, however in these examples, we are using a
single named validator for simplicity. The role of the validator is to propose
blocks which serve a role of ordering transactions. The validator/proposer is
restricted by the fraud proofs constructed in the root blockchain contract. If
they propagate a block with an invalid state transition, any other participant
who receives the block can submit a merkleized fraud proof on the parent
blockchain and the invalid block is rolled back with a slashed penalty.

The blocks are propagated to the participants who wish to observe the blocks,
including participants who hold balances or want to observe/enforce computing on
the individual Plasma chain.

While there is minimal complexity in maintaining deposits of off-chain state,
state transitions and withdrawals create greater complexity.

\subsection{Fraud Proofs}

All states within this child blockchain is enforced via fraud proofs which
allows for any party to enforce invalid blocks, presuming block data
availability.

However, the greatest difficulty in this construction is that there are no
explicit guarantees around data/block availability. 

At the root blockchain (e.g. Ethereum), there are a collection of fraud proofs
which ensure that all state transitions are valid when block data is available.
For complex computation, the state transitions must be merkleized for effective
verification.

Additionally, state transitions can also be enforced via zk-SNARKs/STARKs which
ensure that improper exits are not possible. A zk-SNARKs constructions may need
recursive SNARKs for maximum efficacy, and therefore may require further
research on the possibilities. However, the system is designed to work without
SNARKs.

%Diagram - Plasma-Offchain-Fraud.dia
\begin{figure}[H]
	\makebox[\linewidth]{
		\scalebox{0.75}{
		\includegraphics[width=\linewidth]{figures/Plasma-Offchain-Fraud.pdf}
		}
		}
	\caption{
		Everyone has block data for blocks 1-4. Block 4's committed
		state transition is provably fraudulent via merkleized
		commitments in Block 4 and data from the prior block.
		}
\end{figure}

The fraud proofs ensure that all state transitions are validated. Example fraud
proofs are proof of transaction spendability (funds are available in the
current UTXO), proof of state transition (including checking the signature for
the ability an output can be spent, proof of inclusion/exclusion across blocks,
and deposit/withdrawal proofs. Some more complex proofs require an interactive
game. The general construction would be to take a functional approach towards
block verification. If one programmed this consensus mechanism in Solidity,
there would be an additional input per function of merkle proof of block being
verified and the output would return whether the verification is valid. One
then simply replicates the consensus verification code to process it in compact
merkleized proof form (so that one does not need to process the entire block to
generate fraud proofs).

%Diagram - Plasma-Offchain-Fraud-Penalty.dia
\begin{figure}[H]
	\makebox[\linewidth]{
		\scalebox{0.75}{
		\includegraphics[width=\linewidth]{figures/Plasma-Offchain-Fraud-Penalty.pdf}
		}
		}
	\caption{
		Alice has a copy of all block data, so she submits a proof of
		fraud on the root chain. Block 4 gets invalidated and rolled
		back. The submitter of Block 4 gets penalized by losing a bond
		held in the smart contract. The current block is now Block 3
		(blue). After some set amount of time blocks are finalized and
		no fraud proofs can be submitted. One should only be building
		upon blocks which will not be proven fraudulent by fully
		validating blocks.
		}
\end{figure}

In order for this construction to have minimal proofs, though, all blocks must
provide a commitment to a merkleized trie of the current state, a trie of
outputs spent, a merkle tree of transactions, and a reference to the prior state
being modified.

The fraud proofs ensure that a coalition of participants are not able to create
fraudulent blocks without getting penalized. In the event a fraudulent block is
detected and proven on the root blockchain (or parent Plasma chains), the
invalid block is rolled back. This encourages individual participants to have
incentives against Byzantine behavior, which solves the state transition
vulnerability in federated-peg Bitcoin Sidechains.

The result is highly-scalable state transitions are capable in the Plasma
blockchain while ensuring that observers who have access to block data are able
to prove (and therefore discourage) invalid state transition. In other words,
payments can occur in this chain with only periodic commitments on the root
chain.

\subsection{Deposits}

Deposits from the root chain are sent directly into the master contract. The
contract(s) are responsible for tracking current state commitments, penalization
of invalid commitments using fraud proofs, and processing withdrawals. As
the child Plasma blockchain is a full validator of the root blockchain, the
incoming transaction must be processed using a two-phase lock-in.

Deposits must include the destination chain blockhash to specify the destination
child chain and is achieved using a multi-step process to ensure coins are not
unrecoverable.

%Diagram - Plasma-Offchain-Deposit-Initiate.dia
\begin{figure}[H]
	\makebox[\linewidth]{
		\scalebox{0.35}{
		\includegraphics[width=\linewidth]{figures/Plasma-Offchain-Deposit-Initiate.pdf}
		}
		}
	\caption{
		Alice has an ETH account with 1 ETH. She wants to send it into
		the Plasma blockchain. She sends it into the Plasma contract.
		}
\end{figure}

\begin{enumerate}
	\item
		The coins/tokens (e.g. ETH or ERC-20 token) are sent into the
		Plasma contract on the root blockchain. The coins are
		recoverable within some set time period for a
		challenge/response.
	\item
		The Plasma blockchain includes an incoming transaction proof. At
		this point, the Plasma blockchain is committing to the fact that
		the transaction is incoming and will be spendable in the event
		of either a lock-in transaction or spend initiated by the
		depositor. When this is included, the blockchain is committing
		to the fact that it will honor a withdrawal request. However,
		there is no confirmation yet that the depositor has sufficient
		information to generate a fraud proof, so there is not yet a
		commitment from the depositor. This block includes the addition
		in the state tree, bitmap, and transaction tree, so that there
		is a compact proof of correct inclusion.
	\item
		Depositor signs a transaction on the child Plasma blockchain,
		activating the transaction, which includes a commitment that
		they have seen the block with the chain's commitment in Phase 2.
		The role of this phase is the depositor is attesting to the fact
		that they have sufficient information to withdraw funds. 
\end{enumerate}

After this process, the chain has committed to the fact that they will handle
these coins and gave allocation so that withdrawals can be compactly proven.
With the third phase, the user is attesting to the fact that they can withdraw.

%Diagram - Plasma-Offchain-Deposit.dia
\begin{figure}[H]
	\makebox[\linewidth]{
		\scalebox{0.5}{
		\includegraphics[width=\linewidth]{figures/Plasma-Offchain-Deposit.pdf}
		}
		}
	\caption{
		Alice now has 1 ETH in the Plasma block. She has committed that
		she has seen the funds and is now locked in. The funds are held
		in a smart contract on the root chain, but the ledger record is
		in this particular Plasma blockchain (hence state transitions,
		i.e. sending funds to others or smart contracts) can occur
		without significant root blockchain expense.
		}
\end{figure}

In the event that the depositor has not gone through with Phase 3, then the
depositor can attempt a withdrawal on the root blockchain. The depositor submits
an unconfirmed withdrawal request, and must wait some extra long time period for
anyone on the network to produce a fraud proof that the depositor has signed off
and locked in the funds in the Plasma blockchain. If there is no proof, then the
depositor may withdraw their unconfirmed funds. This withdrawal requires a
sizable root-chain bond to ensure non-Byzantine behavior.

\subsection{Mass Withdrawals and Bitmapped State}

The primary concern around this system is around the inability to verify state. 

To be able to conduct maximum compaction of state transaction, outputs may be
optionally represented in a bitmap. This is necessary for withdrawal proofs
which may be too expensive to conduct on the root chain. The goal of this
construction allows for holding small balances on the Plasma chain. These
balances are held in full reserve on the contract on the root blockchain, but
the full ledger is not on the blockchain. The primary attack which needs to be
mitigated is withheld invalid blocks (with commitments to the root chain). In
the event the system observes invalid state transitions, the participants
conduct a mass exit of the transactions.

With a bitmap construction, a withdrawal includes a bitmap of signed
transactions which wish to exit. A game/protocol is constructed which is
enforced by a smart contract to ensure correct information. The bitmap ensures
that everyone is able to reason about what outputs are being spent.

As this is a bitmap, it necessitates that the state be represented in an unspent
transaction output data structure (UTXO) for maximum efficacy of small-value
balances. Spentness can be compactly proven, and a large set of state
transitions can be cleanly enforced. After a predefined settlement time period,
the bits can be reused.

There is a gradient for expensive with high assurance, to cheap with low
assurance:
\begin{enumerate}
	\item
		Ledger state on root blockchain
	\item
		Ledger state on Plasma, economically viable to enforce a single
		transaction on-chain
	\item
		Ledger state on Plasma, economically viable to enforce using
		bitmap (~1-2 bit cost)
	\item
		Ledger state on Plasma, not economically viable to enforce using
		bitmap on root blockchain. The 1-2 bit mass withdrawal cost too
		high.
\end{enumerate}

For those holding balances which can be enforced on the root chain,
the requirement to conduct a UTXO bitmapped format is not necessary. However,
for those holding balances and enforcement is only viable if the 1-2 bit
transaction gas/fee on the root blockchain is sufficiently low.

For the fourth type (1-2 bit on-chain cost too high for mass withdrawals), the
system is still designed to be resilient (albeit with some assumption that named
entities will be reliable). Later sections in this paper describe a hierarchical
blockchain structure to create many venues where one can economically conduct a
mass withdrawal. Additionally, if the total value of transactions in the fourth
category is significantly below the token value, then it can be game
theoretically too expensive to attack those balances as the tokenholders will
suffer reputational damage.

%diagram of bitmaps

%TODO: Give an example of a state transition, include information about how
%state transitions require timeout to reuse those bits.

\subsection{State Transitions}

By default, state transitions in the Plasma chain run in a similar multi-phase
process as the deposit. This is to ensure that users have information available
to provide state transitions. However, unlike with the deposit construction,
once a transaction is signed and included in a block, there is a commitment to
participate. For this reason, state transitions should include a signature,
state updates (e.g. destination, amount, token, and any other associated state
data), as well as some kind of TTL for expiration and a commitment to a
particular block. This TTL, while not required, should be below the time to
construct exit proofs to ensure that adversarial exit conditions are known.
Pre-signed transactions, of course, should not contain a TTL. Weak liveness
assumptions are presumed with this construction, as there are already liveness
presumptions with withdrawals with regards to deep reorgs. The commitment to a
block is a commitment by the spender that the entity broadcasting the
transaction in the Plasma chain has observed the chain up to that point and is
able to enforce proofs and must be after the block in which the output being
spent has occurred.

The multi-phase commitment occurs as follows for fast finalization:
\begin{enumerate}
	\item
		Alice wants to spend her output in the Plasma chain to Bob in
		the same Plasma chain (without the full transaction record being
		submitted on the blockchain). She creates a transaction which
		spends one of her outputs in the Plasma chain, signs it, and
		broadcasts the transaction.
	\item
		The transaction is included in a block by validator(s) of the
		Plasma chain. The header is included as part of a block in the
		parent Plasma chain or root blockchain, ultimately being
		committed and sealed in the root blockchain
	\item
		Alice and Bob observes the transaction and signs an
		acknowledgement that he has seen the transaction and block. This
		acknowledgement gets signed and included in another Plasma
		block.
\end{enumerate}

For slow finalization, only the first step needs to occur.

After the acknowledgement occurs, the transaction is considered finalized. The
reason the third step exists is to ensure that block availability is ensured
with the participants (Alice and Bob). This third step is not required, but
there will be significant delays in finality. The rationale is that a
transaction should not be viewed as finalized until the block validity and
information availability  can be proven by all parties relevant to the
transaction.

In the event blocks are withheld after step 1, Alice is unclear whether her
transaction has been spent. If a transaction has been included in a block
(whether it is withheld or not), it is treated as unconfirmed if step 3 has not
completed. Therefore, Alice is able to still do a withdrawal of those funds if
she has not signed off on the commit provided her withdrawal message on the
root/parent blockchain occurs before the block is finalized. Alice cannot
withdraw funds after block finalization, and blocks are presumed to be sent to
Bob. If blocks are being withheld before finalization (between step 1 and 2) and
Alice and/or Bob observes this, then Alice may withdraw her unfinalized funds.
If blocks are being withheld after step 2 but before step 3, then it is presumed
that Bob has sufficient information to withdraw funds, but since neither Alice
nor Bob have fully committed to the payment, then it is not treated as complete,
depending on information availability either party may theoretically be able to
claim the funds. If both parties sign off on Step 3, then it is presumed that it
is truly finalized. Pay-to-contract-hash\cite{paytocontract} enforcement occurs
after this step has been completed, specifically when the signatures are
provably observed on-chain. In the event one party refuses to sign or blocks
are being withheld, it is conditional upon a redemption proof. As all states
are eventually committed to the chain via merkle proofs, there is less of a
reliance on pay-to-contract-hash as payments are provable and enforcible after
finalization.

Note that Step 3 can be conditional upon a smart contract instead of a signature
by both parties, i.e. state is conditional upon an HTLC release of a preimage.
This allows for mutli-chain or multi-transaction atomicity. Contract creation
complexity may be increased, and writing higher level languages/tooling around
this may be needed if these features are desired.

\subsection{Periodic Commitments to the Root Chain}

The Plasma chain must be able to create ordering of the blockchain. In a Plasma
chain, there is ordering within blocks, but the blocks are not attested to and
ordered themselves on its own. As a result, it's necessary to create a
commitment on the root blockchain. The Plasma chain publishes its block header
on the root chain and its header is enforced by the fraud proofs. In the event a
fraudulent header is published with data availability for others', any other
participant can publish a fraud proof and the commitment and block is rolled
back, with penalties to the publisher.

These commitments allow for true ordering without equivocation later in time. If
equivocation is attempted, then there is sufficient proof of fraud and can be
penalized. Blocks after a certain time become finalized, and as a result cannot
be reordered provided the root blockchain also reaches sufficient finality.

\subsection{Withdrawals}

Plasma allows one to deposit funds of the native coin and tokens (i.e. ETH and
ERC-20 tokens) off the root blockchain. It additionally allows for state
transitions within the Plasma blockchain whose state is enforced by the root
blockchain provided there is information availability. In the event of
information availability failures, there is a need to do a mass exit on this
Plasma chain. Finally, it's also possible to do a simple withdrawal of funds
held in the Plasma chain.

However, in normal operation, one can do a simple withdrawal.

\subsubsection{Simple Withdrawal}

For a simple withdrawal, one is only allowed to withdraw funds which have been
committed in the root blockchain and ultimately finalized in the Plasma chain.

We have described a design for deposits, compactly representing ledger state,
and state transitions. Up until this phase, other than fraud proofs, there
haven't been any publishing of the current Plasma chain ledger state on the root
blockchain. With withdrawals, though, there needs to be a specific proof that
the funds are held in the Plasma chain and they are current.

Withdrawals are the most critical component, as this ensures the fungibility of
coins between the root blockchain and the child Plasma chains. If one is able to
deposit funds onto the Plasma chain, do state transitions (i.e. transfer coins
to other parties), and those parties capable of withdrawing funds, then the
value should closely map to the value of coins on the root chain. In some cases,
funds on Plasma can be more useful, as it has greater transaction capacity,
while the security is ultimately dependent upon the root chain.

For a simple withdrawal, all funds require a large bond and all withdrawal
requests must include a large bond as a fraud proof. If current block data is
available, then it is possible for a third-party to provide this proof at
exceptionally low cost, as the third-party service can verify the Plasma
blockchain live and ensure that the withdrawal proof is valid.

All participants of the Plasma chain MUST validate all parent Plasma chains and
the root blockchain to ensure that there are no withdrawals in-progress for
particular accounts/outputs when updating state. If a withdrawal is in progress,
a subsequent block cannot spend the coins/tokens, any byzantine behavior here
violates consensus and is subject to fraud proof, penalization, and block
reversal per the Plasma contract in the root blockchain.

A withdrawal occurs in the following steps:
\begin{enumerate}
	\item
		A signed withdrawal transaction is submitted to the root
		blockchain or parent Plasma chain. The amount being withdrawn
		must be whole outputs (no partial withdrawals). Multiple outputs
		may be withdrawn, but they must all be within the same Plasma
		chain. The output bitmap position is disclosed as part of the
		withdrawal. An additional bond is placed as part of the
		withdrawal to penalize false withdrawal requests.
	\item
		A predefined timeout period exists to allow for disputes. This
		is similar to the dispute period in Lightning Network. In this
		case, if anyone can prove an output has already been spent in
		the chain being withdrawn to (in many cases, the root
		blockchain), then the withdrawal is cancelled and the bonded
		withdrawal request is lost. Anyone observing the chain can
		dispute this. If the fraud proof of spent outputs is provided,
		then the bond is lost and the withdrawal is cancelled.
	\item
		A second time delay exists to wait for timeouts of any other
		withdrawal requests with a LOWER block confirmation height. This
		is to force ordered withdrawal in a particular Plasma chain or
		root chain.
	\item
		If the agreed dispute time period defined in the Plasma smart
		contract has elapsed and no fraud proofs are provided on the
		root or parent chain, then it is presumed that the withdrawal is
		correct and the withdrawer will be able to redeem their funds on
		the root/parent chain. Withdrawals are processed in the order of
		old to new in terms of the UTXO/account age.
\end{enumerate}


%diagram showing flow for withdrawal

Note that it is possible, provided that it is economically viable, to do a
withdrawal in the event of a block withholding attack in the Plasma chain.

The fraud proof only requires that anyone on the network prove a duplicate
signature spending from the same output, which can be compactly proven. For
Lightning and other state channels, an additional requirement must prove a
higher nonce as well. For channels, if a lower nonce withdrawal is attempted,
then the funds remain in the Plasma chain, available for withdrawal with a
correct signature. Other constructions are also possible, but design may need to
be front-loaded as part of the creation of the smart contract fraud proofs for
the Plasma chain.

As normal withdrawals are a slow, expensive process, it is likely that they will
be coalesced into a single withdrawal or others are willing to swap coins for
other chains using Lightning or an atomic swap\cite{tiernolan}.

\subsubsection{Fast Withdrawal}

A fast withdrawal is the same construction as the simple withdrawal, but the
funds are being sent to a contract which operates an atomic swap. The funds
being swapped are for funds on the root/parent chain with low timelock for funds
with high timelock exiting the Plasma chain.

A fast withdrawal is not instant. However, it significantly reduces the time to
withdrawal down to the time for transaction finality provided that the Plasma
chain is not Byzantine (including conducting block withholding). For this
reason, a fast withdrawal swap is not possible during block withholding attacks,
and a slow mass withdrawal request would instead be necessary.

A fast withdrawal occurs in the following steps:
\begin{enumerate}
	\item
		Alice wants to withdraw funds to the root blockchain but doesn't
		want to wait. She's willing to pay the time-value for that
		convenience. Larry (the Liquidity Provider) is willing to
		provide this as a service. Alice and Larry coordinate to do a
		withdrawal to the root blockchain. The Plasma blockchain is
		presumed to be non-Byzantine.
	\item
		Funds are locked to a contract on a particular output in the
		Plasma chain. This occurs in a manner similar to a normal
		transfer, in that both parties broadcast a transaction, and then
		later commit that they have seen the transaction in a Plasma
		block. The terms of this contract is that if a contract is
		broadcast on the root blockchain and has been finalized, then
		the payment will go through in the Plasma chain. If no
		transaction proof can be provided, then Alice can redeem the
		funds. It is also possible construct this as an HTLC by having
		Alice generate a preimage, and only release it after she deems
		it acceptable and that the funds are transferred.
	\item
		After the above Plasma block has finalized and Larry is
		confident he can redeem funds in the event the contract
		conditions are met, Larry creates an on-chain contract which
		enables payment to Alice for the specified amount (the amount he
		will receive less the fee he charges for this service)
\end{enumerate}

In our example, Larry the Liquidity Provider must be live and have fully
validated the Plasma blockchain before accepting this swap. If Larry is unable
to fully validate the Plasma chain (or is unfamiliar with the smart contract
fraud proofs defined in the root chain), he should not conduct withdrawals. If
Larry does not want funds in this chain and would instead prefer funds on the
root blockchain, Larry can initiate a withdrawal after this completes, or
conduct the atomic swap as part of a withdrawal itself.

In many cases, it may be more cost effective to do transfers between Plasma
blockchains net-settled with liquidity providers. Transfers can occur between
Plasma chains via Lightning or atomic swaps which allows for rapid finality.

As this is an atomic cross-chain swap, Alice and Larry are not giving custodial
trust of funds to each other. Alice has her funds on the root/parent chain, and
Larry will be able to have full access to the root/parent chain at a later time.
Provided low-cost block availability and finalized non-Byzantine behavior of the
root blockchain, Larry can be reasonably confident that he will receive his
funds even if Larry does not trust the Plasma blockchain itself.

\subsection{Adversarial Mass Withdrawal}

While an adversarial mass withdrawal transaction is within the framework of
Plasma, this is not required for the protocol, its design is primarily for
economic robustness of state (low gas/fees) in the event of block withholding.
If one wishes to use account state inside the Plasma chain, then one can rely
upon other designs as well such as a hierarchy of payments. Additionally, note a
UTXO model is used here, but this system only works well if the root chain uses
an account model. Further, if mass withdrawals are not a necessary or desired
feature, it is possible to use an account model for holding funds in the Plasma
chain and only allow for simple withdrawals (with an incrementing sequence
number).

As the primary consideration of the design of Plasma is in relation to block
withholding attacks to prevent fraud proofs (and other implications of a lack of
data availability), there needs to be mitigations for detected data
unavailability. When block unavailability is detected by users on a Plasma
chain, it is imperative that participants exit the chain by a particular date.
In the event the chain is not exited in time, the result is similar to not
disputing an incorrect withdrawal in Lightning. This mechanism is the key to
correct operation of the Plasma blockchain. Plasma relies upon the fact that if
users detect Byzantine behavior via block withholding, the user is responsible
for exiting the Plasma blockchain. The rationale here is that it is impossible
to detect on the root blockchain whether a block is being withheld (either the
user can assert that they never received the blocks, or the Plasma chain can
assert that the user refuses to recognize that the block is available and is
lying). As a result, the cost for asserted block unavailability has
traditionally been presumed to disclose the current state on-chain (which is
what Lightning does). However, for large blocks and state transitions, this can
be incredibly expensive, and Plasma does not use this construction as it's
unclear who is responsible for paying these costs. Instead, Plasma presumes that
if a user believes that the Plasma blockchain is adversarially withholding
blocks and may impact the ability to enforce state transitions in the future,
then one simply should exit from this Plasma chain to another as quickly as
possible.

Hence, this is defined as an adversarial mass withdrawal insofar as if a block
is unavailable, then it's presumed that the Plasma chain is adversarial or
Byzantine. Mass exit ensures that the Plasma chain's Byzantine behavior does not
impact one's funds beyond a significant time delay and halting the chain.

It is presumed that additional security mitigations using SNARKs will be used in
the future, but the specific design remains an open question. This construction
does not rely upon SNARKs for the withdrawal provided there is periodic liveness
of the observers on the root blockchain, however by enforcing state transitions
within the Plasma chain, the ability for an attacking or Byzantine Plasma chain
to be able to conduct adversarial block withholding to steal funds from those
not periodically observing the Plasma chain can be minimized and enforced by the
security properties of the SNARKs circuit. In that case, it would require a
SNARKs proof to do state transitions and a SNARKs proof for a withdrawal for a
gain of greater assurance of state transitions. However, the goal of Plasma is
not to rely upon SNARKs for correct behavior of state transition provided that
the user is observing the chain, the smart contracts encode the mechanisms
correctly, and is able to withdraw on the root blockchain. Similar benefits can
exist for the Lightning Network in ensuring correct current state by ensuring
off-chain states are only possible in recursive SNARKs proofs committed by a 3rd
party on chains which support smart contracts.

Plasma chains can be secured by defense in depth via the first line of defense
with secure elements/hardware, the second line with SNARKs/STARKs, and the final
line being the interactive game on-chain. The first line may fail, but the
second line secures it using novel cryptography, with the final line being a
public transparent interactive game. We propose Plasma as a system using the
final line initially.

A mass withdrawal is achieved by creating an interactive game whereby exits
occur in the following manner:
\begin{enumerate}
	\item
		Alice coordinates with others on the Plasma chain to conduct a
		mass exit. Multiple mass exits may occur at once, but they
		should not have duplicate withdrawals. In the event they do,
		then the mass exits update their balances, they are processed
		in order, and the person making the duplicate is penalized. All
		parties should coordinate to send their funds directly into
		another Plasma chain.
	\item
		Pat the exit processor is willing to organize this exit. Pat
		coordinates with the destination Plasma chain to send the funds
		and has committed to automatically recognize the funds as
		available in the new chain when the mass exit finalizes.
	\item
		Pat verifies the Plasma chain up to the point of information
		availability. This point must be during the acceptable dispute
		and Plasma finalization period (separate from the root
		blockchain finalization) and is in accordance with the terms of
		the smart contract. Pat shows the pending destination ledger in
		the new Plasma chain to participants. Pat takes all the
		signatures from participants wishing to exit (including in our
		example, Alice). Pat verifies the blockchain that all
		participants have the right to exit for up until the highest
		point of data availability. Pat creates an exit transaction
		with a massive bond (as defined in the root blockchain smart
		contract). Pat may charge a fee for participants exiting.
	\item
		Users sign off on the mass withdrawal again after downloading
		all signatures. This allows for the user to know that Pat will
		not be penalized, and now it is locked in. Users who have not
		submitted the second signature will not have their bit included.
	\item
		Pat then observes whether there are any other exit transactions
		and removes duplicates if necessary, and signs the exit
		transaction and broadcasts on the root blockchain or parent
		Plasma chain. In the event of duplicates, chain parents take
		precedence (up to the root blockchain as the highest
		precedence). Earlier transactions take higher precedence. Upon
		broadcast of the mass exit initiation transaction (MEIT), Pat is
		bonding an attestation that he has the following information
		correct: block validity, UTXO set at block height,
		non-finalization, merkleized mapping of bitmap to UTXO, amount
		committed (in a merkleized sum tree for rapid proofs), Alice and
		others' signatures are available if/when challenged. As part of
		the MEIT Pat publishes a full bitmap of the state being exited.
		This is so that other participants observing the root/parent
		chains are able to verify what is being exited and challenge if
		it looks incorrect. Finalization of the MEIT is a very long
		time, and may take many weeks, hence the MEIT is a last-resort
		transaction (future speedups may be possible with SNARKs).
	\item
		If there are duplicate withdrawals, then Pat has the option to
		update the bitmap and balance being withdrawn during a small
		grace period.
	\item
		Any participant on the network can challenge the data attested
		in the MEIT with a Disputed Mass-Exit Transaction (DMET).
		However, as Pat cannot know if a future block replaces outputs,
		Pat must not be penalized if a transaction has been spent on a
		future block (but a user can be). If a challenge is provided,
		then funds are locked up until the challenge game completes.
		These challenges must occur in an early grace period, and if a
		challenge is valid, then Pat must update the balance which is to
		be withdrawn.
	\item
		If there are no challenges, then after the predefined
		finalization period for the MEIT proceeds and the users receive
		their funds.
\end{enumerate}

The finalization window time for the Plasma chain is the time in which one must
at least periodically watch the chain. After the finalization window, it is
presumed that everyone has Plasma blockchain block data availability up to that
window.

In effect, when the MEIT is created by Pat, Pat is attesting to correct records
up to a particular Plasma block height as well as attesting to the fact that he
has the signatures for withdrawal with each output. Pat is not penalized if an
output has been double-spent after the attested period (as block withholding
should not penalize Pat).

\subsubsection{Mass Withdrawal Dispute: Incorrect Withdrawal Challenge}

In the event a user such as Alice sees that Pat is attempting a mass withdrawal
without her consent, she can invalidate the withdrawal by creating a challenge.

\begin{enumerate}
	\item
		Alice sees that Pat has attempted a mass withdrawal of one of
		her outputs in the Plasma blockchain, as one of the bitmapped
		fields is enabled. Alice broadcasts a challenge with a massive
		bond. This bond is attesting to the fact that a challenge will
		not be produced. She broadcasts this on the blockchain.
	\item
		If the challenge is not disputed after a set time period, Alice
		gets her bond refunded, and the entire MEIT gets cancelled. If
		the challenge is proven to be disputed as Pat or any other party
		produces a fraud proof on her Incorrect Withdrawal Challenge,
		then the MEIT remains in effect and her bond is slashed.
\end{enumerate}

Participants are sure that signatures are available to prove as there is a
second stage in the MEIT (step 4), so they have sufficient information to
dispute a challenge if the challenge is fraudulent. Incentives are against
producing fraudulent challenges, as they will be penalized provided block
availability and non-censorship of the root chain.

\subsubsection{Disputed Mass-Exit Transaction}

In the event that an output has been spent from a Mass Exit Initiation
Transaction in a later block, Pat may not know about this, so he should not be
penalized, as one cannot prove block withholding.

There may be multiple disputes which dispute similar bitmapped sets, but they
all must have a large bond attached.

Any participant can express a bitmap/range of spentness with a large bond. The
large bond is an attestation that a coin has been spent in a later block, with a
commitment to the block header.

However, this dispute cannot be proven compactly, so another iterated challenge
is possible, to issue a Challenge on the Disputed Mass-Exit Transaction (CDMET).

The challenge on this dispute is as follows:
\begin{enumerate}
	\item
		Alice notices that someone (i.e. the operator of the chain doing
		block withholding) attempts to dispute her mass withdrawal she
		is participating in. She submits a challenge on this dispute
		with a large bond attesting to the fact that the submitter of
		the dispute cannot produce a valid spend.
	\item
		The submitter of the dispute must respond to the challenge
		within some time period. If the submitter cannot produce a proof
		of spend, essentially a signature of a later transaction, then
		Alice is vindicated, and the entire dispute gets cancelled (this
		is why duplicate disputes are accepted). If the submitter can
		prove that the coin has been spent, then Alice loses her bond
		and the dispute continues.
\end{enumerate}

\subsection{Recycling UTXOs}

After a spent output has been finalized, it is possible to reuse UTXO bitmaps
for compactness.

\subsection{Summary}

As a result of this mass withdrawal game, it's possible to do a mass withdrawal
which consumes 1-2 bits of information per withdrawal in the most optimistic
case for many participants withdrawing.

Mass exits are something which is necessary in the event of block withholding.
However, this may still be too costly. For this reason, we may need alternative
strategies as well which don't rely on overloading the root chain.

This construction allows for many participants to hold their funds in a child
blockchain, state invalidation occurs via fraud proofs if block information is
available, state transitions can occur (i.e. payments), withdrawals are
available, and mass exits (albeit with some delay) is possible in the event of
block withholding.

\section{Blockchains in Blockchains}

As we have described, Plasma is at its core constructing a method to do scalable
computing, however we need to contend with issues around block withholding to
generate fraud proofs, as well as block space availability. The solution for
block withholding in Plasma is to construct a system whereby one can do mass
exits in the event of chain halt or Plasma block withholding.

However, a mass exit transaction on the blockchain can be very expensive,
especially if the UTXO set is large and the bitmap needs to be published.
Additionally, it may be desirable to just publish a single exit. Mass withdrawal
transactions require a complex interactive game involving many participants. It
should only be used as a last-resort.

Instead, we construct a system of higher and lower courts where particular
venues can exist to prove state. One can view the root blockchain as the Supreme
Court from which the power of all subordinate courts derive their power. It is
the law of the root blockchain which allows for all lower courts to derive
their judicial power. This allows for scalability in venues, it's only when the
state of the lower courts is disputed or halted that one needs to move on to
higher courts for a more represented venue. Broadcasting attestations of state
in higher courts are always possible, but can be more expensive.

All state is merkleized and committed to the root blockchain. In the most
optimistic case, block headers are published in the direct parent chain, and the
parent's chain is published in its parent, and so on until it reaches the root
chain. Inside the header is a merkleized commitment to the blocks one has seen
in the parents.

Transactions can be submitted to the plasma chain and any parent plasma chain,
as well as the root blockchain. The purpose of this is to ensure fungibility and
censorship resistance. Specifically, in the event of halting and non-disclosure
of block movement, one will still be able to withdraw funds.

When a block commitment is submitted, it must wait a certain amount of
confirmations reflected in the root blockchain before it is approved. During
this time, fraud proofs may be submitted to the root blockchain or any
intermediary plasma chain (which is then committed to the root chain via a block
root).

Each individual Plasma chain runs a state machine which bundles commitments into
the Plasma block. The individual Plasma chain may or may not be able to
introspect into the details of the child Plasma chains. Instead, they have a
running confirmed balance of the value of the plasma chain. When a child Plasma
chain updates their state, they submit a hash of their plasma block header to
any of the parent(s) plasma chain or root blockchain.

This means that a particular block state can be submitted to multiple parent
chains. In the event of duplication, then it may not be faulty (but may be
penalized under certain consensus rules depending upon the application). On the
other hand, if there's equivocation in state, e.g. state committed to Parent1
differs from Parent2, then the bonders of the plasma chain may get their deposit
slashed.

A new child state update may update using the following fields in their state
update message: the fee being paid (and denomination), root blockhash being
committed, previous blockhash, parent blockhash being committed to, proof of
deposits, proofs of withdrawal.

Whatever the parent blockchain being committed to assumes that the child chain
has seen everything up to that point, including recursively all parents above
that. This is to commit to proofs that it has is not going to equivocate and
double-spend transactions (and therefore exposing it to slashing in the event of
equivocation).

In the event of equivocation, the parent chain state always takes precedence.
Incentives are created to disclose equivocation by any party knowing them.

Deposits and withdrawals are possible on both parent chains as well as the root
blockchain.

Withdrawals are also possible between plasma chains provided there is sufficient
liquidity and another party is willing to take on the funds elsewhere. This can
be done via a cross-chain atomic swap.

If one wishes to clear using the main blockchain, it could be possible to
construct HTLCs between chains, which look like on-chain Lightning payments.

All fraud proofs must provide a merkleized proof of chain commitments. False
proofs penalize the particular Plasma chain which is responsible for the
fraudulent block.

The primary design complexity is a representation of transaction state broadcast
across multiple parent chains in the interest of censorship resistance. Early
iterations can presume that state transitions/transactions can only be conducted
in the individual Plasma chain, and the only interaction with other chains is
committed message passing to the parent/child and deposits/withdrawals. That
way, the primary complexity is only proofs related to deposits and withdrawals.

Commitments to data are assumed as part of the proofs of inclusion.

\subsection{Receiving Funds Inside a Chain}

In this hierarchical framework of blockchains in blockchains, when one receives
funds from another user, the process is as follows if Alice wants to send funds
to Bob in a Plasma chain 3 levels deep:
\begin{enumerate}
	\item
		Alice coordinates with Bob that she wants to send funds to Bob.
		Alice discloses to Bob the Plasma chain in which Bob will
		receive the funds. Bob decides whether to accept the payment,
		specifically, Bob should ensure that the smart contract on the
		root blockchain is one in which he is willing to accept payment
		(smart contract code/mechanisms, as well as acceptable consensus
		exit delays, etc.)
	\item
		If this is payment for some good, they pre-sign a statement
		defining the conditions of payment, in many cases it will be a
		proof of payment by block inclusion in the blockchain with
		sufficient maturity, however it can also be a
		pay-to-contract-hash in some circumstances. This is not on-chain
		but merely to attach the terms for settlement to prove to
		others.
	\item
		Alice makes the payment inside the Plasma chain. The block gets
		signed off by the validators, and the commitment to the block
		header gets published into the parents' blocks. Merkleized
		commitments to child Plasma chains are included in every parent
		block and ultimately included in the root blockchain.
	\item
		Bob fully syncs with the root blockchain, then validates the
		chain in which the funds are being received and any parent of
		that. Bob does not need to validate other Plasma chains which
		his funds are not a part of. Bob can fully validate that Alice
		has made the payment in the Plasma chain with sufficient
		maturity in the worst-case. However, if rapid finality is
		desired, Alice can sign off on the payment being fulfilled in
		the new block (see the previous statement on receiving payments
		inside a Plasma chain). If Alice is willing to sign off on
		the payment and Bob accepts it (as he is able to prove a
		withdrawal), then it is presumed that finality is reached. Bob
		is able to withdraw the funds from this Plasma chain.
\end{enumerate}

The key aspect of this design is that one is wholly responsible for validating
the child blockchains. If Bob does not validate the Plasma chain and all parents
(ultimately the periodic commitment into the root chain is published), then it
should not be treated as fulfilled. Similar to the construction in Lightning
Network, Bob does not need to care what happens in the other Plasma blockchains.
He only observes the correctness of the chains which matters to him. When he has
the ability to use the coins, he then is confident he is able to spend them.

\subsection{Receiving Funds from a Parent Chain}

Receiving funds from a parent chain is similar to a deposit from the root
blockchain, the only difference being the recipient needs to validate all parent
Plasma chains (rather than just the Plasma chain itself). Deposits into a child
Plasma chain is fast.

\subsection{From a Tree to a Web}

While the above description is about a single parent chain, it's possible for
plasma chains to watch multiple root blockchains. This allows for one to update
balances with child chains. Care must be given as failure on one parent may not
be recognized by all participants at once, and cascading systemic failures must
be mitigated via time-delays and minimizing assumptions of cross-chain
liquidity. The correct construction for this is an open problem.

\subsection{Mitigating the Block Withholding Problem}

By constructing many venues where one can broadcast a withdrawal transaction,
there can now be many possible venues to exit from a chain which has halted or
has blocks withheld. If a child chain fails, then an individual simple exit can
be processed on the parent chains, even if transactions become expensive on the
root chain.

This allows one to have some measure of confidence in holding micropayment
outputs on the Plasma chain, provided that they have assurance that one of the
parent Plasma chains are operating correctly. This goal is the primary cause of
this, as well as mitigating the impact of cascading failures.

If one has sufficiently large output balance held, they do not need to do
significant underwriting if there isn't significant time-value, however, if one
holds a single low-value output (where paying the transaction fee becomes too
expensive), then one should have some measure of assurance that one of the
parent Plasma chains have availability. If one wants greater assurance, one can
run nested chains deeply with many independent parties running each Plasma chain
at each level. There are some tradeoffs that exist by doing it this way, though,
as if a particular Plasma chain starts becoming Byzantine, then everyone will
need to do a mass-withdrawal to a new chain. If there is a parent which is not
byzantine, though, it is possible to continue operating and facilitate rapid
transitions to another chain if the parents refuse to process the Byzantine
chain's commitments.

It's possible that services may arise whereby it doesn't do anything other than
process transactions in the event of a child chain fails. The operator of this
service doesn't need to do anything unless a child chain fails (to the point
where they can be sufficiently passive that they could hypothetically turn off
their servers until a failure occurs, block headers automatically skip them to
be broadcast on a chain a level above the passive operator).

We expect that many of the withdrawals in parent chains will be simple
withdrawals instead of mass withdrawals, as a parent chain can have incredibly
high transaction volume (block size/gas limit).

\subsection{Exiting}

Mass exits are possible to a parent chain or the root chain. If the child chain
begins to act Byzantine, it's presumed that any state may be invalid, similar to
a Plasma chain without nested parent chains. Similarly, mass exits
are a way to quickly exit from a Byzantine parent chain. It is possible to skip
a parent chain (or the child chain itself) to its parent(s) or the root chain.

While it may seem as though there's some complexity in the design, the
presumption is that if any chain is Byzantine, all its children must act. There
are optimizations that are possible so that exits are possible without
coordination via a heartbeat (exits are by-default without a signature revoking
it on the users' side and the Plasma chain itself making a commitment that it
has received it, but may be a premature optimization).

The construction is basically the same as the simple exit or mass exit, however
there are some minor changes in the design to support nested chains. Exits may
be duplicated, but exits on the parent chain always take priority. If a parent
chain begins to act Byzantine, then the exit can be committed on the root chain
as well. It is the responsibility of the (perceived to be Byzantine Plasma
chain) to reflect and update their state of the parent/root chain duplicated
exit and revoke the duplicated exit in its own chain. However, if it does not do
so, users' funds will be available on the root chain.

If a parent is Byzantine but the child in which one holds funds is operating
correctly, it is possible to avoid doing a complex mass exit transaction.
Participants find a new chain to send their funds to and do a simple exit
whereby a liquidity provider receives funds in this child chain and the other
users receive funds on the new chain (without the Byzantine parent). Commitments
of the child chain blocks are published on the root chain or the higher-level
parents (avoiding the Byzantine node). Users quickly have funds in the new chain
and the liquidity provider exits their funds onto the root or highest parent
later. The purpose of this is that new funds can be quickly allocated into the
new chain and exits occur rapidly.

\subsection{Scalability}

This allows for UTXO bitmap scalability, in the event that a bitmap gets too
large, one just needs to split the bitmap into multiple child chains. For child
chains, it is presumed that it is represented as an account balance with block
height nonces (and candidate chain tips) instead of an output. Similarly, for
states which prefer to use accounts instead of UTXOs, then it is possible as
well provided that one is willing to make the tradeoff of only supporting simple
withdrawals.

The end result of this is a great deal of scalability for users. They only need
to observe Plasma chains which their funds are held (as well as its parents).
This effectively shards the dataset into validation which affects oneself.

\section{Plasma Proof-of-Stake}

We propose a simple proof-of-stake construction. This is likely not the optimal
proof-of-stake construction, however this is illustrative of what is possible in
Plasma chains.

Up until now we've assumed that the operator of a Plasma chain is a single
entity who is responsible for signing off on the blocks. If they create an
invalid block, anyone else who has the block data can generate a fraud proof and
roll back the block with penalties to the operator. This proof is possible as
the operator has signed off on the block with their signature. The publication
of a merkleized commitment of the Plasma block in the root chain (and as the
highest parent Plasma block includes commitments to its children's state
updates), hence state updates are ordered and bonded to correct behavior.

However, in many cases it's preferable to construct a proof-of-stake chain
instead of a single-party proof-of-authority chain. This minimizes risks related
to block withholding (it's possible to get the best of both words by nesting a
chain into a single-party proof-of-authority as well as an open multi-party
proof-of-stake chain). A tokenized proof-of-stake chain also gives incentives
for tokenholders to operate correctly, as the value of the token declines from
Byzantine behavior. More detail on the potential value of tokenization is
provided in a later section.

Proof of Stake construction is easier to construct in Plasma, as it still relies
upon the robustness of the underlying root blockchain. Problems related to
withholding, finality, and other factors are pushed to the reliability of the
root chain. Plasma can only at best be as secure as the root chain. If the root
chain is running Proof of Work, then this is Proof of Stake on Proof of Work
(Plasma on the root blockchain, respectively). If the root blockchain is Proof
of Stake, this construction is Proof of Stake on Proof of Stake, however the
Proof of Stake mechanisms may be simpler or different than the one running on
the root blockchain.

\subsection{Nakamoto Consensus Incentives}

We attempt to replicate the primary incentives from the Nakamoto Consensus
(Proof of Work mining). One of the most important incentive to replicate is that
of encouraging block propagation to other miners.

Many existing proposed proof-of-stake mechanisms rely upon leader election,
where at time $t_{0}$ a leader is elected and at time $t_{1}$, the leader has
the right to produce a block. This does not replicate the Nakamoto Consensus
incentives around block propagation. The Nakamoto Consensus does do leader
election, but rather it does probabilistic leader election. If one finds a
block, one believes it is likely that one is the leader, but one isn't
completely sure. Someone else could have mined a block at the exact same moment.
The best way to maximize odds that one is the leader is to broadcast that block
as far and wide as quickly as possible so others can build on top of it. This
creates incentives for information availability.

Plasma's Proof of Stake construction needs to do something similar.

We made tradeoffs in that we want to encourage everyone to propagate their
blocks as far as wide as possible. There may be other constructions (especially
ones in which construct heavy reliance on random selection and probabilistic
leader election after-the-fact by allocating random scores to particular
branches and determining the chain tip as the branch with the highest score).

\subsection{Example Simple Proof of Stake Model}

While this is a simple proposal to construct a Proof of Stake model, it is
likely that it isn't anywhere near optimal. The goal is to construct something
simple which Plasma can use.

Instead of creating enforcement mechanisms, the approach is to simply create
incentives for proper coordination and correct behavior (block propagation).

Fees are allocated and distributed by the root contract and paid out
periodically if desired, but the accounting is done inside the chain itself.

As part of the staking contract, stakers' funds allocated assign towards a
delegated staker. The staker is responsible for acting on behalf of the user and
the user gets penalized if the staker is faulty. Staking is committed for a
specific time (e.g. 3 months). The minimum amount per staker is one percent of
all tokens, with a maximum cap of five percent. If one wants to allocate more
than five percent, then they should use multiple staking identities (the purpose
is to maximize data distribution and minimize the efficacy of below 51\%
cartels).

Funds get allocated depending on whether the past 100 Plasma blocks are
representative of all participants. For example, if someone is staked at 3
percent of the stakers, they should be 3 percent of the previous 100 blocks. If
it is above that amount, the individual staker receives no additional reward for
publishing extra block commitments. If the past 100 blocks, there is below 3,
then the current block creator receives fewer rewards. Only one block can get
allocated per block on the root chain.

This encourages all participants to coordinate and include everyone's blocks
equally. The presumption is that one does not need to construct enforcement
mechanisms, the participants will coordinate and ensure some kind of scheme
(e.g. round robin) to ensure maximum rewards.

If they do not receive the maximum in transaction fees due to improper amount of
blocks, the funds get allocated to a pool to pay out future blocks.

The result is economic encouragement that one includes participation from
everyone.

However, this is not yet complete, as we are only encouraging accurate
participation from stakers. In every block is a merkleized commitment to data in
random parts of blocks from the past 100 blocks. This forces the staker to have
the full block data and consequently forces a block creator to propagate it
to all stakers.

The chain tip is determined by maximum reward, if there are parallel branches,
then the one that wins is the one that has the maximum fee reward from maximum
coordination.

This construction is not designed to stop 51\% attacks, but instead is designed
to encourage block propagation (as the threat is equal if one commits to
withheld blocks). Additionally, this construction is reliant upon information
availability and impartiality in block inclusion on the root chain; it is not
possible to construct this type of Proof of Stake on the root chain due to
assumptions on data availability and censorship incentives.

\section{Economic Incentives}

Within a Proof of Stake validation model, it's possible to construct incentives
which align with correct operation of the contract terms. While fidelity bonds
ensure accuracy towards the chain, we need to create further incentives around
data availability and discourage halting. By only allowing staking using a
token specific to the Plasma chain, one is able to ensure that there is
incentive to continue operating, as the value of the token is derived from the
net present discounted value of all future returns from staking. Consequently,
network failures reduce the value of the tokens being held and individual
actors have heavy incentive to act in the best interest towards the network's
continued operation.

The operators of the Plasma chains receive fees from broadcasting transactions
on-chain. Computation may charge different fees for different operations, and
fees may cascade down, especially for complex operations. While there is
incentive for more transactions to be conducted in deep child chains, it's
possible to create commitments for funds transferred or computation in the child
chain, which propagates up. This allows parent chains to charge for computation
in child chains, and if there is an incorrect commitment, the block data is
invalid and unenforceable. This is not a necessity, and in many cases, one would
prefer more computation on child chains with less necessity for distributing
fees up to parent chains.

For systems upgrades, it's possible to upgrade the system by creating another
contract which accepts the same token and announcing a transition period (or the
community collectively decides in decentralized systems).

This may create systems which run themselves. Whereas one needed to pay and
operate services on cloud computing to host sites which do data storage and
computation, it is now possible to construct a set of smart contracts (with
fraud proofs), a token, and with sufficient number of participants paying fees,
that the system can operate on its own with a set of stakers who continue to
correctly operate the network and computing infrastructure, truly making
computing on the amorphous cloud.

\subsection{Tokens vs. Coins and Economic Security}

These fraud proofs and bonds ultimately held on the root blockchain can be the
native token, e.g. Ether (ETH) for Ethereum, or it can be a separate token
maintaining the consensus rules of the underlying blockchain.

It is ostensibly most simple to use the root blockchain's native token (e.g.
ETH), however there are interesting economic security implications.

If the goal is to prevent chain halt and faulty behavior, depending on the
blockchain application, there may be insufficient incentive to prevent faulty
behavior if only ETH was used. The token's value will decline if the chain
halts or is Byzantine. Additionally, the token value is approximately the NPV of
future transaction fees which can make the token valuable. If one staked ETH,
then one is staking the value derived from the time-value of the bond relative
to the amount of fees gained. It's expected that this value being bonded would
be much lower than the net present discounted value of the token. Additionally,
chain halts and block withholding are difficult to prove and discourage and if
one bonds ETH and gets the money back after the staking period, there is
insufficient incentive to act in non-Byzantine ways, whereas with tokens, the
value of the token would decline with widespread Byzantine behavior.

\section{MapReduce for the Blockchain}

Nearly anything which is computable on MapReduce should be computable on this
chain as well. This requires a deep restructuring of the way we think about
computation and programming on the blockchain. This is MapReduce but with fraud
proofs. Each node represents a blockchain. This is highly compatible with the
Plasma blockchain tree structure described in a previous section.

E.g. if one wants to do the standard word count, you can create a merkle tree of
chains operating a reduce function. If there is proof of fraud, then the node
that produced the fraud gets penalized. If you can produce a reduce function on
summation, then you can also produce an average. E.g. average prices, etc. Map
function is just sending computation to the individual chains and then
committing to the results. Obviously, there is constraint around data
throughput, which is why reduce fraud proofs are necessary. Arbitrary
computation of all types is not possible, but it is possible to resolve many
types of problem sets; usually memory constrained problem sets can be solved by
first running sorting algorithms, which makes tradeoffs of inter-Plasma chain
traffic.

If the nodes cannot produce the actual blocks to prove computation, then their
results should be discarded and rolled back. Note that this does not guarantee
computational scalability that MapReduce does (since you need to observe the
chain to maintain consensus), however, it does give enforcement of activity and
the ability to scale it up for the actors. As a result, the primary limitations
are around the fact that parties which are affected by particular computation
should observe that set of computation. If one only needs to observe some small
part, then it should be fine, but if one needs to observe all computation then
it does not offer scalability benefits (only benefits around scalable
assurance). That said, many problems can be solved this way, e.g. decentralized
exchange (your mapped set cares about your own trades, if everyone else's netted
execution is enforcible you don't care about the details), etc.

The block format must be compatible with data which can be computable in a
TrueBit construction. There are commitments to state (to be able to construct a
UTXO/state trie which allows for proofs of state transition on
inclusion/exclusion), account trie (for child chains and complex state
transitions), commitment on fees (tree which commits to state transition on
fees), merkleized transactions, commitment to data passed from parent/child
blocks, commitment to seen parent/child blocks (to prevent reordering), as well
as any business logic (e.g. word count example would have a merkleized sorted
commitment to words and where it was seen). By constructing merkle commitments,
one can create smart contracts provable on root or parent chains which are able
to prove incorrect state transitions. There are some problem sets which may not
be compatible with this format, but generic computation without significant
memory requirements are possible. A mental framework for this is to treat the
maximum memory size for computation as equivalent to the maximum of data allowed
in a fraud proof.

A series of map and reduce functions allow for the blockchain to operate in such
a way whereby there is obligation to process data. This requires the parent and
children to create obligation of processing. Children must include the parent
passing in data else the chain will halt. The parent can enforce computation in
children and if the children halts, enforcement of computation can occur by
broadcasting the data in the parent chain and attesting to the proof there. A
primary threat in TrueBit constructions is related to the issues around halting,
so care should be constructed to allow for continued operation if the child
chain halts, although this requires a great deal of complexity, especially over
time (datasets can change and having time consistency is more difficult to
reason about for some problems).

By constructing blockchain computation in a map and reduce framework with child
chains, it's possible to take existing computer science research and directly
apply it to distributed systems problem sets which exists for blockchains. It is
possible construct Solidity contracts which produce many useful business
applications in a scalable manner. One only needs to compute and verify activity
relevant to oneself.

\section{Example Applications}

Decentralized Applications can be reframed as a MapReduce problem with economic
incentives for correct activity bonded by a token.

\subsection{Reddit Clone on the Blockchain}

This is primarily about data storage (CRUD). Primarily computation and proofs
are around access control, identity (votes and posts), and moderation. Many web
applications are actually just doing CRUD on the back end.

The root blockchain contains the smart contract consensus rules and fraud
proofs. The topmost parent contains accounts of subreddits. Each subreddit is a
Plasma blockchain child of the topmost parent. Within each subreddit is a Plasma
chain of posts. That child chain of posts contains comments as well. Consensus
mechanisms enforce access control. Randomized commitments to previous block data
(with random nonces provided by the parent chain) are committed to in every
block header. A reduce function periodically is computed for top posts and other
statistics.

An individual user's computer downloads the data and software local to the
machine formats the data. Submitting data requires paying transaction fees to
incentivize inclusion of data, may have a fee for downloading old block data
depending on availability.

To view a specific post, the user verifies the commitment on the root chain,
then goes to the chaintip on the topmost parent (back n blocks to go back to
finality time period, perhaps a week's worth), finds the state account trie for
the relevant subreddit. Connects to a DHT network to discover nodes on the
subreddit, downloads the chaintip (plus back n blocks to verify) of the
subreddit and view list of posts, download the state trie as a light client and
raw data for the relevant post with comments. Users only need to observe parts
of the Plasma chains which are relevant only to themselves (download only the
posts and subreddits relevant to themselves).

This is a simple example of data storage with some computation on the
blockchain. It is possible that validators fully validate all nodes, however,
it's possible to shard it out. Once the sharding is too far, though, there are
information availability concerns. One way to mitigate that is to give full
control of the child chain to the subreddit owner.

\subsection{Decentralized Exchange}

A reddit clone on the blockchain, while makes obvious the implications for CRUD
web applications, doesn't take significant advantage of MapReduce operations
with the exception of site statistics.

A decentralized exchange shows that it is possible to trade low-latency for high
computational capacity. As there are many states, it's possible outputs are
defined in accounts instead of UTXO, or for each step in the state machine that
a larger bitmap is used to represent each state instead of a single boolean
value representing spentness in the bitmap.

Similar to the subreddit, there is a tree of child chains each representing a
trading pair. Within each is a tree of chains to maximize scalability (for low
activity pairs it may only be one Plasma chain, but for high-activity chains it
may have much more children). Each of these chains have bonded activity and the
amount which is able to transact per round is limited by the bonded amount.

The first step is to have balances in a child chain, so this is like a payments
Plasma chain as a base.

Next, orders are issued directly into the child chain. As part of the commitment
to the parent, all orders are coalesced into a merkleized commitment of a single
order book represented as a single order by the chain itself. This step
recursively reduces all children's order books into a single orderbook
represented by that chain, until it reaches the highest Plasma chain parent.
After orders are received, the order window closes and the trades execute in a
batch fashion.

After this reduce step completes and is committed into the root blockchain, the
individual chains are notified of their allocation, via a map step. The parent
tells the child what allocation of their orders filled in a deterministic
manner. Provided the child was able to see the other orders (which implies it is
able to observe the parent chain during this step), they will be able to prove
correct execution of allocation in this map step. After receiving the
allocation, the map step continues recursively to that chain's children.

When this completes, a final reduce step is made by committing all fund updates
into a block and submitting the block header to one's parent.

There are further optimizations by allowing for multiple MapReduce rounds in the
event of significant price movements (allowing for high accuracy in pricing),
however this general construction enables immensely high volume. It is
theoretically possible to conduct all of the world's trading activity in this
framework, with some tradeoffs in speed by turning it into a batch execution
exchange fully committed to and bonded by the root blockchain.

This type of construction is useful for many types of financial activity and
computation.

\subsection{Decentralized Mail}

To create D-Mail, it's possible to represent one's account in a Plasma chain and
require payment to receive mail (insert message in the chain). Submissions are
encrypted with one's public key. Enforcement is possible to ensure that unknown
entities must pay, further optimizations are possible using zk-SNARKs. Parent
chain contains a directory of chains and enforces payment. Simple design.

\subsection{Decentralized CDN}

It's possible to have a decentralized CDN. Similar construction as Ethereum
sharding proposal. Treat each child blockchain as a shard. Have a randomness
beacon (can be root blockhash or something else). Shuffle the data around
between shards every n blocks. Parent chains are responsible for commitments to
the shuffle. Others may lazily keep archives. Data loss is announced and those
who keep archives are rewarded. Incentive to propagate, since one only gets
rewarded if the other shard has the data. Robustness depends upon the
requirement for the "length" of the flow; the more robust, the more shards need
to have a copy at any one time. The key insight here is that storage is a
function of bandwidth. The data is not treated as stationary data on disk. All
data is actually in flow and motion to its next destination; very Taoist
approach.

For downloading the data, one verifies the parent chain's shard and random
beacon to know which shard contains the data, then connect to it via a DHT
identifying the peers and download the data.

\subsection{Private Chains}

Participants are not obligated to disclose the data in the chain to others
(although there is nothing stopping it from being public), and as a result if
participants on the chain want to have some private blockchain network enforced
by the root chain, they can do so. This is analogous to the intranet/internet
separation. Transactions may occur on the local private chain, but also
communicate and have financial activity bonded by the public chain.

\section{Attacks, Risks, and Mitigations}

\subsection{Smart Contract Code}

Writing good smart contract code is hard. The security is entirely reliant upon
correct execution of the fraud proofs. It is possible that some fraud proofs
may not be included and invalid state transitions may be valid in the root
chain.

\subsection{Closing Transaction on Main Chain Too Expensive}

There's a risk where a transaction can be closed on the main chain, but it's not
economically feasible to do so. This can create certain types of exit scams
whereby one coordinates a large amount of small values to add up to a large
value.

This can be mitigated by having an exit provision, sorting all transactions by
the exit provision and allowing an exit from an entire chain at once after a
certain dispute mediation period. This additionally allows for a 3rd party
watcher to watch on one's behalf. However, this construction increases
complexity significantly. These pre-signed coalesced transactions can propagate
onto other parent networks depending on what part of the chain fails, or
ultimately on the root network. However, this also relies upon named parties
acting correctly, which is why individuals should coalesce all unspent payments
to a single output or set of outputs whereby exiting transactions are
economically feasible.

Additionally, one can leave micropayments on chains which are bonded by a
highly valued token, and piggyback on that value (as there is sufficient
disincentive from completely killing the Plasma chain).

If generalized recursive SNARKs/STARKs become feasible, it would be
theoretically possible to ensure that the entity withdrawing doesn't have
authority to do unauthorized exits, even with withheld blocks.

\subsection{Finality}

The dispute window for exit is essentially creating finality assumptions. If the
underlying chain has some significant bonded costs to reorgs to force finality,
then it can significantly reduce risks around deep chain reorgs creating a lack
of synchronicity between chains. An example of mitigations is the planned
Ethereum CASPER finality gadget.

\subsection{Root Chain Lack of Capacity or Increasing Costs} 

Without mitigations in place, in the event that fees or gas becomes too
expensive, it would not possible to exit the transaction within a set time
period. E.g. if transaction fees / gas costs increase 50-fold or there's
insufficient space to exit transactions and miners do not increase the capacity
or gas limit.

There are several mitigations by extending the exit delay by pausing the exit
counting mechanism which allows for exits in an orderly fashion. This can be
done by pausing exits so long as there is an exit transaction which occurs in
the past x blocks. This allows everyone to get out over time provided that
there's at least one exit transaction recently. If an exit occurs before one's
exit, then the counter is reset. The result would be that it is undetermined how
long one must wait before receiving funds back, which increases the expense of
liquidity providers rates. A simple mechanism would be to pause the clock on
withdrawal times if average blockchain gas/fee costs are above a particular very
high amount (gated by some reasonable upper bound time in which this paused
state can persist).

Users holding funds should ensure that at least one parent Plasma blockchain has
a high degree of confidence with information availability (ideally multiple
independent parents).

\subsection{Root Chain Censorship}

The design assumes that over 51\% of the root chain is honest. If participants
on the root chain coordinate to attack the network by censoring blocks, then
significant difficulties may arise from enforcement of exit transactions or
state updates, potentially creating significant issues around loss of funds.
Censorship is the primary factor of security and value-at-risk as well as
finality tricks (e.g. CASPER finality gadget) may need to be constrained in
future chains.

This can be mitigated by adding in zk-SNARKs/zk-SNARKs proofs of funds, but
requires significant new engineering and research.

The use of a very large bond as part of every exit transaction encourages fraud
proofs as the miners will likely be rewarded a significant majority of that
fraud proof and censorship would be highly discouraged.

The security afforded to this network is a function of the honesty and
correctness of one's parent plasma chains, the root blockchain, and the balance
size.

Systemic limitations globally on the amount transferred (velocity limits on
exits per block) and ensuring it is lower than a finality gadget could be a
possible path of mitigation.

\subsection{Chain halt}

If the chain halts, after a set time period can have a pre-committed state
transition offer. The consuming chain taking over the transactions can then
broadcast an acceptance of that and the entire chain moves. This is only
allowed after a chain (ignoring transactions broadcast on parents) does not
move forward for a set time period. Fraud proofs can dispute the chain tip.

There may be greater incentive to chain halt for financial activity involving
complex state transitions.

\subsection{Inability to Change Consensus Rules}

As the design is front-loaded it's not possible to change consensus rules
without pre-programming in that ability. This can be mitigated by creating
upgrade paths as part of the system (e.g. forced halt after a certain date).
This inability can also have social implications around the inability to halt
chains, as there is economic incentive for the tokenholders to continue
operating the system, so may become difficult to halt the Plasma chain after it
launches.

\section{Future Research}

There are additional areas of future research including benefits related to
security around these chains. A current area of research is generalized
recursive SNARKs/STARKs, which would significantly boost the security of exit
transactions. It could still be desirable to have defense in depth, hence the
last line of defense would be a straightforward decentralized exit mechanism
allowing for disputed proofs, with front lines being novel cryptography and
secure hardware elements. Further developments in novel uses for pairing
cryptography or other forms of homomorphic encryption may also be helpful.

Greater specificity is required on the ability to watch multiple root chains at
once while remaining synchronized (beyond simply just forcing hard
synchronicity).

Further research is needed around finality and its interaction across multiple
chains, as well as further minimization of blockchain exit risks (SNARKs/STARKs
could help here).

\section{Conclusion and Summary}

Plasma is design with a primary focus on ensuring information availability
(especially with regards to block withholding attacks) with data compression.

We propose a mechanism whereby one can submit enforcible commitments which allow
one to hold funds in a blockchain whose state is enforced by the root
blockchain.

This allows for significant computation and storage across a wide amorphous
network of computers. Activity is bonded by economic actors enforcing the
commitments, which is ultimately enforcible across all parent chains and
enforcement flows down to the root blockchain who holds the smart contract
enforcing truth. This construction allows for one to do state transitions which
would not be otherwise cost-effective on the root chain.

It is possible from this construction for a blockchain to be processing
commitments on nearly all financial computation worldwide (as long as it doesn't
take too much working memory at one time). It is only if there is invalid
computation that proofs are submitted and those commitments rolled back. One
does not need to give custodial trust to the operators of the chain.

To reduce the incentives around chain halting and other Byzantine behavior, fees
create incentive for the chain to continue operating. Halting is discouraged if
the Plasma chain is bonded by activity in a token specific to this chain. If the
chain halts, the value of the chain declines, creating significant economic
incentive for continued operation.

This incentive and structure allows for one to create decentralized autonomous
programs which continually operate funded by transaction fees. These
decentralized autonomous applications can create true cloud computing, whereby
the data is being processed and validated, but whose member set is ever
changing and amorphous. Plasma can allow blockchains to scale up to serve
generalized applications with the number of users without significant
limitations. An application creator can just write the smart contract code,
then submit the code to the blockchain and the incentives can persist to
continue operating these contracts' computation as long as people use the
Plasma chain to pay its fees. 

\section{Acknowledgements}

Many thanks go out to the TrueBit authors for a design and implementation of
merkleized proofs, including Chrstian Reitwie{\ss}ner for review. Thanks to
Vlad Zamfir for many areas of inspiration and his general framing of ideas
helpful in formalizing these ideas. Thanks to Thomas Greco, Piotr Dobaczewski
and Pawe\l{} Peregud
%and Emin Gun Sirer 
for feedback and contributions.

TODO: Ask others for acknowledgements.

TODO: Improve bibliography and get more citations

TODO: Finish diagrams

%OLDER WITHDRAWALS ARE PRIORITIZED, PROCESSED FIRST within a dispute window
% Must have maturity, then dispute window, then wait for all transactions
% before dispute window to time out, then process all transactions (DOES NOT
% CASCADE) This solves the issue of block withholding sufficiently.
%NEED TO TAG MESSAGE TO BLOCKHEIGHT, priority goes to lowest height. Higher up
%the stack are prevented from using lower heights. Withdrawal signer can
%prioritize or de-prioritize based on which chain. Deeper in the stack gets
%precedence in DISPUTES/DOUBLE-SPENDS. IOW, priority goes lower in stack
%normally, but in a dispute priority goes higher in the stack.

%Faulty nodes via censorship/halting the entire plasma chain results in longer
%withdrawal times, as they cannot do a withdrawal proof, so liquidity providers
%may not be confident they have the funds correctly. However, if they have data
%before finality and it gets into a block before then, they can be confident.

%\section{Design Assumptions}
%depends on CASPER finalization
%exit scam risk, dust getting stuck 
%peer discovery of individual chains are presumed to be available via DHT. If
%chains are unavailable, then users should not receive funds on that chain.
%assumes root chain is censorship resistant

%\section{Appendex A: Doing this with Bitcoin}
%
%Make a full validation extension block as a soft-fork, and then do whatever you
%want. Hard code it to only do payments on Plasma with fraud proofs. Done.

%bibliography
\bibliographystyle{unsrt}
\bibliography{plasma}

\end{document}
